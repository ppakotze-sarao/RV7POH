\chapter{Emergency procedures}
\thispagestyle{fancy}
\minitoc[n] % Creating an actual minitoc

\section{Introduction}
This section provides checklist and amplified procedures for coping with emergencies that may occur.  Emergencies caused by airplane or engine malfunctions are extremely rare if proper preflight inspection and maintenance are practised.  

\section{Airspeeds for Emergency Operations}
\begin{table}[H]
\caption{Airspeed for Emergency Operations}
%\label{tab:airspeed_emergencies}
  \begin{tabularx}{\linewidth}{
    |>{\hsize=0.2\hsize}X| 
     >{\hsize=0.6\hsize}X|
     >{\hsize=0.2\hsize}X| 
} 
 \hline
   & Description &  Airspeed\\ 
   \hline
 $V_{ref}$ & Engine Failure After Take Off & 70 kt\\ %manoeuvre
  \hline
 $V_{glide}$ & Best glide & 78 kt  \\ 
   \hline
 $V_{a}$ & Manoeuvring speed & 123 kt \\ %manoeuvre
 \hline
\end{tabularx}
\end{table}


\section{ENGINE FAILURE}
\subsection{Engine Failure during Takeoff Run}
\begin{enumerate}[(1)]
  \item Throttle -- CLOSED
  \item Brakes -- APPLY
  \item Wing Flaps -- RETRACT
  \item Mixture -- IDLE CUT OFF
  \item Ignition -- OFF
  \item Master -- OFF
\end{enumerate}

\subsection{Engine Failure After Take Off}
\begin{enumerate}[(1)]
  \item Airspeed -- 78 KIAS
  \item Mixture -- IDLE CUT OFF
  \item Fuel Selector -- OFF
  \item Ignition -- OFF
  \item Flaps -- AS REQUIRED
  \item Master -- OFF
\end{enumerate}

\subsection{Engine Failure During Flight}
\begin{enumerate}[(1)]
  \item Airspeed -- 78 KIAS
  \item Fuel Selector -- SWITCH TANKS
  \item Mixture -- RICH 
  \item Fuel pump -- ON 
  \item Ignition -- ON
\end{enumerate}

  If power not restored

\begin{enumerate}[(a)]
\item Ignition  -- Cycle OFF then ON
\item Alternate Air -- PULL
\item Throttle and Mixture -- RESET
\end{enumerate}

 If power not restored perform Forced Landing.

\section{FORCED LANDING}
\subsection{Emergency Landing without Engine power}
\begin{enumerate}[(1)]
  \item Airspeed -- 78 KIAS
  \item Fuel Selector -- OFF
  \item Mixture -- IDLE CUT OFF
  \item Ignition -- OFF
  \item Flaps -- AS REQUIRED
  \item Master -- OFF
  \item Canopy -- UNLATCH
\end{enumerate}

\section{FIRES}
\subsection{During Start on the Ground}
\begin{enumerate}[(1)]
  \item Ignition -- START continue cranking
  \item Mixture -- IDLE CUT OFF
  \item Fuel Selector -- OFF
  \item Ignition -- OFF  
  \item Master -- OFF  
  \item Airplane -- EVACUATE
  \item Fire Extinguisher -- DISCHARGE into cowl outlet
\end{enumerate}

\subsection{Engine Fire in Flight}
\begin{enumerate}[(1)]
  \item Mixture -- IDLE CUT OFF
  \item Fuel Selector -- OFF
  \item Master -- OFF  
  \item Cabin Heat -- OFF
  \item Airspeed -- SELECT glide speed to extinguish fire
\end{enumerate}
 Prepare for emergency landing.

\subsection{Electrical Fire in Flight}
\begin{enumerate}[(1)]
  \item Master -- OFF  
  \item Air Vents -- CLOSED
  \item Cabin Heat -- CLOSED
  \item Extinguisher -- DISCHARGE 
\end{enumerate}
After fire stopped open air vents to clear cabin.  Return power to essential instruments only if safe to do so.

\section{ELECTRICAL POWER SUPPLY FAILURES}
\subsection{Ammeter Shows Battery Discharge}
\begin{enumerate}[(1)]
  \item Alternator -- Cycle OFF for 15s then ON
\end{enumerate}
If battery discharge continues reduce electrical load and terminate flight as soon as possible.

\section{Amplified Procedures}
\subsection{Engine Failure}
If an engine failure occurs during take off run, the most important thing is to stop the aircraft on the remaining runway.

The first response to an engine failure after takeoff is the prompt lowering of the nose to maintain airspeed in the glide.   The checklist procedures assume that adequate time exists to secure fuel and ignition systems before touchdown.

After an in flight engine failure establish best glide speed first.  Should an engine restart fail a forced landing without power must be completed.

\subsection{Spins}
Van's Aircraft does not consider spins to be a recreational aerobatic manoeuvre.  Accidental spins can result from a variety of conditions in which asymmetric wing lift is induced.  Spins normally are caused by improper rudder usage coupled with a stall. 

Should a spin occur, the following recovery procedure is suggested:
\begin{enumerate}[(1)]
  \item Throttle -- IDLE
  \item Rudder -- OPPOSITE to rotation
  \item Ailerons -- NEUTRAL
  \item Control -- FORWARD enough to break stall\\
Hold these controls until rotation stops.
  \item Rudder -- NEUTRALISE then gently recover from dive
  \end{enumerate}