\chapter{Limitations}
\thispagestyle{fancy}
\minitoc[n] % Creating an actual minitoc

\section{Introduction}
This section includes operating limitations, instrument markings and basic placards necessary for the safe operation of the aircraft, its engine, standard systems and equipment.

\section{Airspeed Limitations}
\begin{table}[h]
\caption{Airspeed Limitations}
\label{tab:airspeed_limits}
%\begin{tabular}{ |p{0.5in}|p{2.5in}|p{0.5in}|p{2in}| } 
  \begin{tabularx}{\linewidth}{
    |>{\hsize=0.08\hsize}X| 
     >{\hsize=0.4\hsize}X|
     >{\hsize=0.12\hsize}X| 
     >{\hsize=0.5\hsize}X| 
  }
   \hline
  &SPEEDS &KIAS &REMARKS\\ 
 \hline
 $V_{NE}$ & Never exceed speed & 200 kt & Do not exceed this speed in any operation\\ 
 \hline
 $V_{NO}$ & Maximum structural cruising speed & 168 kt & Do not exceed this speed except in smooth air, and then only with caution\\ 
  \hline
 $V_{A}$ & Manoeuvering Speed & 123 kt & Do not make full or abrupt control movements above this speeds \\ 
  \hline
%  {$V_{FE}$}  & \shortstack[l]{Maximum Flap Extended Speed: \\To 20$^{\circ}$ Flaps\\ 20$^{\circ}$ - 40$^{\circ}$ Flaps} & \shortstack[l]{\\96 \\ 87} & Do not exceed these speeds with the given flap settings \\ 
% \hline
  {$V_{FE}$}  & Flap Extended Speed: \newline To 20$^{\circ}$ Flaps \newline 20$^{\circ}$ - 40$^{\circ}$ Flaps & - \newline 96 kt \newline 87 kt & ~\newline Do not exceed these speeds  with the given flap settings \\ 
 \hline
\end{tabularx}
\end{table}

\section{Airspeed Indicator Markings}
\begin{table}[H]
\caption{Airspeed Indicator Markings}
\label{tab:airspeed_indicator}
  \begin{tabularx}{\linewidth}{
    |>{\hsize=0.15\hsize}X| 
     >{\hsize=0.25\hsize}X|
     >{\hsize=0.6\hsize}X| 
} 
 \hline
  MARKING & KIAS VALUE or  RANGE &  SIGNIFICANCE \\ 
 \hline
 White Arc & 51 kt - 87 kt& Full flap operating range.  Lower limit is maximum weight $V_{S0}$ in landing configuration.  Upper limit is maximum speed permissible with flaps extended.\\ 
 \hline
 Green Arc & 56 kt - 168 kt & Normal operating range.  Lower limit is maximum weight $V_{S}$ with flaps retracted.  Upper limit is maximum structural cruising speed.\\ 
 \hline
 Yellow Arc & 168 kt - 200 kt& Operations must be conducted with caution and only in smooth air. \\ 
 \hline
 Red Line  & 200 kt & Maximum speed for all operations \\ 
 \hline
\end{tabularx}
\end{table}

\section{Power plant limitations}
Engine Operating Limits for Takeoff and Continuous Operations:

\begin{tabularx}{\linewidth}{
    >{\hsize=0.5\hsize}X
    >{\hsize=0.5\hsize}X
  }
 Engine manufacturer: & Avco Lycoming\\  
 Engine Model number: & IO-360-A1B6\\
 Maximum Power: & 200 hp\\
 Maximum Engine Speed: & 2700 RPM\\
 Maximum Cylinder Head Temp: & 500$^{\circ}$F \textit{(260$^{\circ}$C)} \\
 Maximum Oil Temperature: & 245$^{\circ}$F \textit{(118$^{\circ}$C)}\\
% Oil Pressure: & 25 to 95 psi\\
 Oil Pressure: & 125 psi\\
 Fuel Pump Pressure: & -2 - 35 psi \\
 Fuel Injector Pressure: & 14 - 45 psi \\
 Propeller Manufacturer: & Hartzell\\
 Propeller Model No: & HC-C2YR-1BFP/F7497-2\\
 Propeller Diameter: & Maximum 72"\\
 Propeller Blade Angles: & Low: 13.6$^{\circ}$\newline High: 35$^{\circ}$\\
\end{tabularx}

\section{Power plant instrument markings}
Power plant instrument marking and their colour code significance are shown in Table \ref{tab:eng_limits}.  The limits tabled are based on the Lycoming engine operating manual. To maximise engine service life green arc limits should be adhered to and engine power settings of 65\% or less is recommended.

\begin{table}[H]
\caption{Power Plant Instrument Markings}
\label{tab:eng_limits}
\begin{tabularx}{\linewidth}{
    |>{\hsize=0.24\hsize}X| 
     >{\hsize=0.18\hsize}X|
     >{\hsize=0.19\hsize}X| 
     >{\hsize=0.20\hsize}X| 
     >{\hsize=0.19\hsize}X|  
     } 
\hline
\multirow{2}*{Instrument}      & Minimum Limit &Normal Operating & Caution Range &  Maximum Limit \\
\hline
& RED LINE &GREEN ARC & YELLOW ARC & RED LINE \\
% \hline
%& Minimum Limit & Normal Operating & Caution Range & Maximum Limit \\ 
% \hline
%  Instrument & RED LINE & GREEN ARC & YELLOW ARC & RED LINE\\ 
 \hline
%  Tachometer & \dotfill & 2100 - 2500 RPM & \dotfill & 2700RPM \\ 
  Tachometer & \dotfill & 1800 - 2500 RPM & \dotfill & 2700RPM \\ 
  \hline
  Manifold \newline Pressure & \dotfill & 15 - 25 \newline in. Hg & \dotfill & 28.7 \newline in. Hg \\ 
 \hline
  Oil \newline Temperature & 140$^{\circ}$F & 165 - 220$^{\circ}$F & \dotfill & 245$^{\circ}$F \\ 
 %  Oil \newline Temperature & 140$^{\circ}$F \newline \emph{60$^{\circ}$}C & 165$^{\circ}$-220$^{\circ}$F\newline  \emph{74$^{\circ}$-104$^{\circ}$C}  & \dotfill & 245$^{\circ}$F \newline \emph{118$^{\circ}$C}  \\ 
 \hline
  %Cylinder Head Temperature & 150$^{\circ}$F(66$^{\circ}$C)  & 150$^{\circ}$F(66$^{\circ}$C) to 400$^{\circ}$F(205$^{\circ}$C)  & 400$^{\circ}$F(205$^{\circ}$C) to 435$^{\circ}$F(224$^{\circ}$C) &  500$^{\circ}$F(260$^{\circ}$C)  \\ 
Cylinder Head \newline Temperature & 150$^{\circ}$F& 150 - 400$^{\circ}$F  & 400 - 435$^{\circ}$F &  500$^{\circ}$F \\ 
%Cylinder Head \newline Temperature & 150$^{\circ}$F\newline \emph{66$^{\circ}$C} & 150$^{\circ}$-400$^{\circ}$F \newline \textit{66$^{\circ}$-205$^{\circ}$C} & 400$^{\circ}$-435$^{\circ}$F \newline \textit{205$^{\circ}$-224$^{\circ}$C}&  500$^{\circ}$F\newline 260$^{\circ}$C \\ 
 \hline
% \hline
  Fuel Pressure (Flow) & -2 psi & -2 - 35 psi & \dotfill & 35psi \\ 
  \hline
  Oil Pressure  & 25 psi & 55 - 95 psi & \dotfill & 95/115psi\\ 
 \hline
\end{tabularx}
\end{table}

\section{Weight Limits}
  \begin{tabularx}{\linewidth}{
    >{\hsize=0.5\hsize}X
     >{\hsize=0.5\hsize}X
  }
 Maximum Takeoff Weight: & 1800 lbs  \\ 
 Maximum Landing Weight: & 1800 lbs  \\  
 Maximum Aerobatic Weight: & 1600 lbs  \\  
 Maximum Weight Baggage: & 100 lbs  \\  
\end{tabularx}

\section{Centre of Gravity limits}
Reference Datum: 70" forward of the leading edge of the wing. 
\subsection{Normal Operations}
Center of Gravity range:
\begin{itemize}
\item{Forward:} 78.7" aft of datum at 1800 lbs or less
\item{Aft:} 86.8" aft of datum at 1800 lbs or less
\end{itemize}

\subsection{Aerobatic Operations}
Center of Gravity range:
\begin{itemize}
\item{Forward:} 78.7" aft of datum at 1600 lbs or less
\item{Aft:} 84.5" aft of datum at 1600 lbs or less
\end{itemize}

\section{Manoeuvre limits}
This aircraft is designed for flight in both the normal and aerobatic category.
\subsection{Normal Category}
The normal category is applicable to aircraft intended for non-aerobatic operations.  These include manoeuvres incidental to normal flying, stalls (except whip stalls), lazy eights, chandelles and steep turns in which the angle of bank is not more than 60$^{\circ}$.

\subsection{Aerobatic Category}
The aerobatic category is applicable to the aircraft when loaded within limits.  Aerobatic manoeuvres include, Loops, Horizontal Eights, Immelman Turns, Aileron Rolls, Barrel Rolls, Snap Rolls, Vertical Rolls and Split S.

\section{Flight load limit factor limits}
\subsection{Normal Category}
At any weight between 1600lbs and 1800lbs or at any CG location aft of 84.5 inches, the aircraft flight load limits are +4.4 and -2.2G.

\subsection{Aerobatic Category}
At an Aerobatic gross weight of 1600lbs the airframe structure is designed to withstand indefinitely flight load limits of +6 and -3G.  

\section{Kinds of operation limits}
This aircraft is approved for any operation approved in accordance with the current Authority to Fly.

\section{Fuel limitations}
  \begin{tabularx}{\linewidth}{
    >{\hsize=0.4\hsize}X
    >{\hsize=0.6\hsize}X  }
Approved Fuel Grades: & 100/130 Aviation Fuel (Blue).\\
Capacity: & 42 US Gal (159$\ell$)\\
Useable fuel: & 40 US Gal (151$\ell$)\\
\end{tabularx}

\section{Placards}
The follwoing information is displayed in the form of composite or individual placards.

\begin{enumerate}[(1)]
\item At fuel valve (at appropriate locations):

  \begin{tabularx}{\linewidth}{
    >{\hsize=0.4\hsize}X
    >{\hsize=0.6\hsize}X  }
Fuel Total & 40 US Gal (151$\ell$)\\
Left Tank & 20 US Gal \\
Right Tank & 20 US Gal \\
\end{tabularx}

\begin{comment}
\item In full view of the pilot:
\begin{table}[h]
\caption{Aerobatic Entry Speeds}
\label{tab:aerobatic_speeds}
  \begin{tabularx}{\linewidth}{|
    >{\hsize=0.5\hsize}X|
    >{\hsize=0.25\hsize}X|
    >{\hsize=0.25\hsize}X|
  }
\hline
Manoeuvre: & Speed [mph] & Speed [kt]\\
\hline
Loops, Horizontal eights:&140 - 190 mph & 122 - 165 kt\\
\hline
Immelman Turns: & 150 - 190 mph & 130 - 165 kt\\
\hline
Aileron Rolls, Barrel Rolls: &120 - 190 mph & 104 - 165 kt\\
\hline
Snap rolls &80 - 110 mph & 70 - 96 kt\\ 
\hline
Vertical Rolls: &180 - 190 mph  & 156 - 165 kt\\
\hline
Split-S:       &100 - 110 mph & 87 - 96 kt\\
\hline
\end{tabularx}
\end{table}
\end{comment}

\end{enumerate}